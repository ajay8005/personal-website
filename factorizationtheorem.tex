

\documentclass{article}

\usepackage[margin=1.2in]{geometry}
\usepackage{graphicx}
\usepackage{amsmath,amssymb,amsthm,bm}

\newtheorem{theorem}{Theorem}

\begin{document}

\begin{theorem}[Factorization Theorem]
Let $(\mathcal{X}, \mathcal{A})$ be a measurable space, and let $\mathcal{P} = \{P_\theta : \theta \in \Theta\}$ be a family of probability measures on $(\mathcal{X}, \mathcal{A})$. Suppose there exists a $\sigma$-finite measure $\mu$ on $(\mathcal{X}, \mathcal{A})$ such that each $P_\theta$ is absolutely continuous with respect to $\mu$, and let $p_\theta(x) = \dfrac{dP_\theta}{d\mu}(x)$ denote the Radon-Nikodym derivative.

A statistic $T: \mathcal{X} \to \mathcal{Y}$ is sufficient for $\theta$ if and only if there exist measurable functions $u: \mathcal{Y} \times \Theta \to [0, \infty)$ and $v: \mathcal{X} \to [0, \infty)$ such that for all $x \in \mathcal{X}$ and all $\theta \in \Theta$,
\[
p_\theta(x) = u(T(x), \theta) \, v(x).
\]
\end{theorem}

\begin{proof}
We will prove both directions of the equivalence.

\textbf{($\Rightarrow$) Sufficiency implies factorization:}

Assume that $T$ is sufficient for $\theta$. By the definition of sufficiency, the conditional distribution of $X$ given $T(X) = t$ does not depend on $\theta$. Formally, for any measurable set $B \in \mathcal{A}$ and for all $\theta \in \Theta$,
\[
P_\theta(X \in B \mid T(X) = t) = P(X \in B \mid T(X) = t),
\]
where the right-hand side is independent of $\theta$.

Consider the conditional density of $X$ given $T(X) = t$, which we denote by $k(x \mid t)$. Since this density does not depend on $\theta$, we can write
\[
p_\theta(x) = P_\theta(X = x) = P_\theta(T(X) = t, X = x) = P_\theta(X = x \mid T(X) = t) \, P_\theta(T(X) = t).
\]
However, in continuous settings, we need to be careful with densities. Instead, we can use the following approach.

Let $f_{T, \theta}(t)$ be the marginal density of $T(X)$ under $P_\theta$, and let $k(x \mid t)$ be the conditional density of $X$ given $T(X) = t$, which does not depend on $\theta$ due to sufficiency. Then,
\[
p_\theta(x) = f_{T, \theta}(t) \, k(x \mid t).
\]
Since $k(x \mid t)$ does not depend on $\theta$, we can let
\[
v(x) = k(x \mid t),
\]
and since $t = T(x)$, $v(x)$ is a function of $x$ alone.

Now, define
\[
u(t, \theta) = f_{T, \theta}(t).
\]
Therefore, we have
\[
p_\theta(x) = u(T(x), \theta) \, v(x).
\]

\textbf{($\Leftarrow$) Factorization implies sufficiency:}

Assume that there exist measurable functions $u$ and $v$ such that
\[
p_\theta(x) = u(T(x), \theta) \, v(x).
\]
We need to show that $T$ is sufficient for $\theta$.

Consider any measurable set $B \in \mathcal{A}$. We aim to show that the conditional distribution of $X$ given $T(X) = t$ does not depend on $\theta$.

First, compute the conditional probability density of $X$ given $T(X) = t$ under $P_\theta$:
\[
P_\theta(X \in B \mid T(X) = t) = \frac{\int_{B \cap T^{-1}(t)} p_\theta(x) \, d\mu(x)}{\int_{T^{-1}(t)} p_\theta(x) \, d\mu(x)}.
\]
Substitute $p_\theta(x) = u(T(x), \theta) \, v(x)$:
\[
P_\theta(X \in B \mid T(X) = t) = \frac{\int_{B \cap T^{-1}(t)} u(t, \theta) \, v(x) \, d\mu(x)}{\int_{T^{-1}(t)} u(t, \theta) \, v(x) \, d\mu(x)}.
\]
Since $u(t, \theta)$ is constant with respect to $x$ over $T^{-1}(t)$, it can be factored out:
\[
P_\theta(X \in B \mid T(X) = t) = \frac{u(t, \theta) \int_{B \cap T^{-1}(t)} v(x) \, d\mu(x)}{u(t, \theta) \int_{T^{-1}(t)} v(x) \, d\mu(x)}.
\]
The $u(t, \theta)$ terms cancel out:
\[
P_\theta(X \in B \mid T(X) = t) = \frac{\int_{B \cap T^{-1}(t)} v(x) \, d\mu(x)}{\int_{T^{-1}(t)} v(x) \, d\mu(x)}.
\]
This expression is independent of $\theta$, which means that the conditional distribution of $X$ given $T(X) = t$ does not depend on $\theta$. Therefore, $T$ is sufficient for $\theta$.
\end{proof}

\end{document}