\documentclass{article}

\usepackage[margin=1.2in]{geometry}
\usepackage{graphicx}
\usepackage{amsmath,amssymb,amsthm,bm}
\usepackage{latexsym,color,minipage-marginpar,caption,multirow,verbatim}
\usepackage{enumerate}
\usepackage{times}

\newcommand{\cP}{\mathcal{P}}
\newcommand{\cX}{\mathcal{X}}
\newcommand{\cR}{\mathcal{R}}
\newcommand{\cF}{\mathcal{F}}

\newcommand{\EE}{\mathbb{E}}
\newcommand{\PP}{\mathbb{P}}
\newcommand{\QQ}{\mathbb{Q}}
\newcommand{\RR}{\mathbb{R}}
\newcommand{\ZZ}{\mathbb{Z}}

\newcommand{\td}{\,\textrm{d}}

\newcommand{\simiid}{\overset{\text{i.i.d.}}{\sim}}
\newcommand{\ep}{\varepsilon}

\newcommand{\red}{\color{red}}
\newcommand{\blue}{\color{blue}}
\definecolor{darkblue}{rgb}{0.2, 0.2, 0.5}


\newcommand{\sol}{~\\\color{darkblue}{\bf Solution:~\\}}

\title{Stats 210A, Fall 2023\\
  {\bf Optional} Homework 0\\
  {\large {\bf {\em Not} Due on}: Wednesday, Aug. 30}}
\date{}

\begin{document}

\maketitle

Lecture 1 included a ``whirlwind tour'' of measure theory at the heuristic level that we'll be using in class. Problem 1 is meant to give a little more intuition about densities and the others are meant to motivate measure-theoretic probability a bit.

Note this problem set has three problems; a typical problem set will have 5.

%{\red Corrections and added hints in red.}
%{\blue Solutions in blue.}


\begin{description}

\item[1. Densities]\hfill\\

For a given point $x \in \cX$, the {\em Dirac measure} is defined as 
\[
\delta_x(A) = 1\{x \in A\} = \begin{cases} 1 & \text{ if } x \in A\\ 0 & \text{otherwise}\\ \end{cases}.
\]
\
Essentially, $\delta_x$ is the measure that puts a unit of mass on $x$ and none anywhere else.\footnote{In a sense this is defined identically to the indicator function $1_A(x)$, but we think of one as being a function of $x$ with $A$ fixed, and the other as a function of $A$ (a measure) with $x$ fixed.} Integrals wrt $\delta_x$ are defined as $\int f(u) \td\delta_x(u) = f(x)$.

 Furthermore, suppose $\mu_1$ and $\mu_2$ are both measures on $\cX$, and $a_1,a_2 \geq 0$. You may use without proof that the sum $\nu = a_1\mu_1 + a_2\mu_2$ is also a measure, and that for ``nice enough'' functions,
\[
\int f(x)\td \nu(x) = a_1\int f(x)\td \mu_1(x) + a_2\int f(x)\td \mu_2(x).
\]

\begin{enumerate}[(a)]
\item Let $x_1,x_2,\ldots, x_n$ be integers (not necessarily all distinct), and define two measures on the set $\ZZ$ of all integers: the counting measure $\#$ from class, and the {\em empirical distribution}
\[
\widehat{P}_n(A) = \frac{1}{n}\sum_{i=1}^n \delta_{x_i}(A).
\]
That is, $\widehat{P}_n(A)$ is the fraction of points that fall into the set $A$.

{\bf Note:} if $x_1,\ldots, x_n$ are sampled from some distribution $P$ then $\widehat{P}_n$ is a natural nonparametric estimator of the measure $P$.

Show that $\widehat{P}_n$ is absolutely continuous with respect to $\#$ but not the other way around. What is the density of $\widehat{P}_n$ with respect to $\#$? Is it possible to define a density of $\#$ with respect to $\widehat{P}_n$?



\item For $\cX = [0,\infty)$, define the measure $\mu(A) = \lambda(A) + \delta_0(A)$, where $\lambda$ represents the Lebesgue measure. For fixed $\theta \in \RR$, define the random variable 
\[
X = \max(0,Z) \text{ where } Z \sim N(\theta, 1),
\]
what is the density of $X$'s distribution with respect to $\mu$? 


\item Consider two densities $p_1$ and $p_2$ with respect to some common measure $\mu$ on a space $\cX$ (not necessarily the same $\mu$ from part (b)). Suppose $p_1$ and $p_2$ both result in the same measure $P$ defined by $P(A) = \int 1_A(x)p_i(x)\td \mu(x)$.

Define the set $A = \{x:\; p_1(x) \neq p_2(x)\}$, and show that $\mu(A) = 0$ ({\bf Hint:} consider sets like 
\[A_{n} = \left\{x:\; p_1(x) - p_2(x) \in \left[\frac{1}{n+1}, \;\frac{1}{n}\right)\right\}\]
for $n=1,2,\ldots$.)

Don't worry about whether the measure is well-defined for $A_n$ (i.e., whether these sets are measurable). They are in the sense we need them to be.


\end{enumerate}

\item[2. A conditional probability paradox]\hfill\\

Let $X,Y \simiid N(0,1)$. This problem is meant to show that by carelessly conditioning on probability-zero events we can get ourselves into trouble. It is directly inspired by a calculation I personally flubbed a few years ago.

\begin{enumerate}[(a)]
\item Defining $S = X + Y$ and $D = X - Y$, show $S$ and $D$ are independent and conclude that
\[
\EE[X^2 + Y^2 \mid D] = D^2/2 + 1
\]




\item Now define the polar parameterization $(R,\Theta)$ with $R = \sqrt{X^2 + Y^2}$ and $\Theta \in [0,2\pi)$ such that $X  = R\cos \Theta$ and $Y = R\sin \Theta$. Show that $R$ is independent of $\Theta$ and conclude that
\[
\EE[X^2 + Y^2 \mid \Theta] = 2
\]



\item Use (a) and then (b) to find the expectation of $X^2 + Y^2$ conditional on the event $X = Y$. Can you come up with an intuitive explanation for how we could have arrived at two different answers?


\end{enumerate}

{\bf Moral:} Intuition may fail us when we condition on a measure-zero event, and in cases like this the meaning can be ambiguous and give different answers. Conditioning on a random variable, on the other hand, tends to give less ambiguous answers (there are still some ambiguities, similar to those we encounter in defining densities, but they don't really matter).



\item[3. Non-measurable sets]\hfill\\

This problem goes through a construction of a non-measurable set, meant to motivate measure theory from a real analysis perspective. It concerns the impossibility of defining ``volume'' for every subset of the unit interval $U=[0,1)$.

For $x,y \in \RR$ define the ``wraparound addition'' (modulo 1) as the fractional part of their sum:
\[
x \oplus y = x + y - \lfloor x + y \rfloor.
\]

Recall that for $x \in \RR$ and $A \subseteq \RR$ we define the set $x + A = \{x + a:\; a \in A\}$. Analogously, we can define 
\[
x \oplus A = \{x \oplus a:\; a \in A\} \subseteq U
\]

Any reasonable definition of ``volume'' on the interval should have several properties:
\begin{enumerate}[(i)]
\item Additivity: $\lambda(\bigcup_{i=1}^\infty A_i) = \sum_{i=1}^\infty \lambda(A_i)$ if all $A_i\subseteq U$ and $A_i \cap A_j = \emptyset$ for all $i \neq j$.
\item Translation invariance: $\lambda(x \oplus A) = \lambda(A), \;\forall x\in U, A\subseteq U$.
\item Interval length: $\lambda([x,y)) = y - x, \;\forall 0 \leq x \leq y \leq 1$.
\end{enumerate}
Assume that some measure $\lambda$ exists which satisfies (i)--(iii) and which is defined for all subsets of $U$. We will go through several steps to derive a contradiction.

\begin{enumerate}[(a)]
\item Define the function $A(x)$ mapping elements of $U$ to subsets of $U$, via $A(x) = x \oplus \QQ$, where $\QQ$ is the set of rational numbers. Show that $\lambda(A(x)) = 0$ for any $x$.


\item Consider the range $\cR_A = \{A(x):\; x\in U\}$. Show that $\cR_A$ is a collection of uncountably many subsets of $U$, all of which are disjoint from each other. That is, show that for any $x,y\in U$, we have either $A(x)=A(y)$ or $A(x) \cap A(y) = \emptyset$.


\item Now, let $B\subseteq U$ denote a new set, which we construct by selecting a {\em single element} from each set $R\in\cR_A$ (it doesn't matter which element; note this step uses the axiom of choice.)

Define a new function $C(x) = x \oplus B$ and define $\cR_C = \{C(x):\; x \in \QQ\}$. Show that $\cR_C$ is a collection of {\em countably} many subsets of $U$, all of which are disjoint from each other, and whose union is $U$.

\item Show that no matter what value $\lambda(B)$ takes, $\lambda$ will have to violate one of the properties (i)--(iii) ({\bf Hint:} what does the value of $\lambda(B)$ imply about $\lambda(U)$?)
\end{enumerate}

Because the Lebesgue measure satisfies properties (i)--(iii), it follows that $\lambda$ must not be defined for every subset of $U$.

{\bf Moral:} One motivation (but not the only motivation) for the idea of a $\sigma$-field is to exclude pathological counterexamples like this.


\end{description}


\end{document}
